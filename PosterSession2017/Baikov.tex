\documentclass[10pt]{article}
\usepackage[usenames]{color} %used for font color
\usepackage{amssymb} %maths
\usepackage{amsmath} %maths
\usepackage[utf8]{inputenc} %useful to type directly diacritic characters
\begin{document}
\textbf{\textcolor{blue}{Baikov parametrization}}

\textcolor{red}{Baikov parametrization} is essentially a coordinate change $(\bold{q} \mapsto x:=D)$, which allows us to write a scalar Feynman integral in the following form:
\begin{eqnarray}
\nonumber
\int d^d \bold{q} \frac{1}{D_1^{n_1} \cdots D_N^{n_N}}
=
C \int \frac{dx_1 \cdots dx_N}{x_1^{n_1}\cdots x_N^{n_N}} P^{\frac{d-M-1}{2}}
\end{eqnarray}
where $M = L+E$ and $P$ is the Jacobi determinant of the coordinate change.
We call this $P$ \textcolor{red}{Baikov polynomial}, which is a polynomial of $x$'s and external kinematics.


\end{document}