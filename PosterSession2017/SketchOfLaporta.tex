\documentclass[10pt]{article}
\usepackage[usenames]{color} %used for font color
\usepackage{amssymb} %maths
\usepackage{amsmath} %maths
\usepackage[utf8]{inputenc} %useful to type directly diacritic characters
\begin{document}
\textbf{\textcolor{blue}{Sketch of a variant Laporta algorithm}}

We represent an IBP identity as a descending list of terms:
\begin{eqnarray}
\nonumber
0 = \sum_i c_i I_i \Leftrightarrow \verb|[c1*I1,c2*I2, ..]|
\end{eqnarray}
where $I_1$ is the most \textcolor{red}{complicated} integral.
Among the system of IBP identities, we group the equations with the same highest integral, and place like the following form:
\begin{eqnarray}
\nonumber
\left(\begin{array}{cccccccc} 
\bullet & \bullet & \bullet \\
\bullet & \bullet & \bullet & \bullet &\bullet \\
\bullet & \bullet & \bullet & \bullet &\bullet & \bullet\\
\bullet & \bullet & \bullet & \bullet &\bullet & \bullet & \bullet\\
&\bullet & \bullet & \\
&\bullet & \bullet & \bullet & \\
&\bullet & \bullet & \bullet &\bullet &\bullet &\bullet & \bullet\\
&&\bullet & \bullet & \bullet \\
&&\bullet & \bullet &\bullet & \bullet &\bullet \\
&&&\vdots
\end{array}\right)
\end{eqnarray}
I.e., in each subset of equations, they share the highest integral (the left most $\bullet$) and they line up from shorter to longer.

Within the subset of equations, substituting the top element into the successors, we eventually have the \textcolor{red}{right triangular form}:
\begin{eqnarray}
\nonumber
\left(\begin{array}{cccccccc} 
\bullet & \bullet & \bullet \\
&\bullet & \bullet & \bullet & \bullet &\bullet \\
&&\bullet \\
&&& \bullet & \bullet & \bullet &\bullet & \bullet\\
&&&&\bullet & \bullet & \\
&&&&&\vdots
\end{array}\right)
\end{eqnarray}

Once we reach this right triangular form, use \textcolor{red}{back substitution}; for the first two lists, using second equation, we substitute it in the first and represent the highest integral via lower integrals:
\begin{eqnarray}
\nonumber
\left(\begin{array}{cccccccc} 
\bullet & \bullet & \bullet \\
&\bullet & \bullet & \bullet & \bullet &\bullet 
\end{array}\right)
\mapsto
\left(\begin{array}{cccccccc} 
\bullet & 0 & \bullet & \bullet & \bullet &\bullet \\
&\bullet & \bullet & \bullet & \bullet &\bullet 
\end{array}\right)
\end{eqnarray}
Finally we have the following form of equations:
\begin{eqnarray}
\nonumber
\left(\begin{array}{cccccccc} 
\bullet & 0 & 0&0&0 & \star & \star & \star \\
&\bullet & 0 & 0 & 0 & \star & \star & \star \\
&&\bullet &0 &0 & \star & \star & \star \\
&&& \bullet & 0 & \star & \star & \star \\
&&&&\bullet & \star & \star & \star \\
&&&&&\star & \star & \star\\
\end{array}\right)
\end{eqnarray}
The $\star$'s are the \textcolor{red}{Master Integrals} of this system.



\end{document}