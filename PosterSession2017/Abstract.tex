\documentclass[10pt]{article}
\usepackage[usenames]{color} %used for font color
\usepackage{amssymb} %maths
\usepackage{amsmath} %maths
\usepackage[utf8]{inputenc} %useful to type directly diacritic characters
\begin{document}
\textbf{Abstract}

\textcolor{red}{Scattering amplitudes} in quantum field theories allow us to compare the phenomenological prediction of particle theories with the measurements at collider experiments.
The study of scattering amplitudes, in terms of their symmetries and analytic properties, provides a framework to develop techniques and efficient algorithms to evaluate cross sections.

In this poster, we describe an interesting technique based on the generation of \textcolor{red}{Integration By Parts identities} for the reduction of a generic E+1 legs L loops integral toward the sum of a minimal basis of Master Integrals.
We also introduce \textcolor{red}{Baikov parametrization} which translates IBP identities into polynomial forms from integrals.

\begin{center}
\hspace{-0.5cm}{\small \textit{This work was done under the supervision of Prof. A. Ferroglia and Prof. G. Ossola}}
\end{center}


\end{document}