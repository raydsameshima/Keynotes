\documentclass[10pt]{article}
\usepackage[usenames]{color} %used for font color
\usepackage{amssymb} %maths
\usepackage{amsmath} %maths
\usepackage[utf8]{inputenc} %useful to type directly diacritic characters
\begin{document}
\textbf{\textcolor{blue}{Laporta algorithm}}

For a fixed $N$, IBP identities among the family of integral $I(\bold{n})$ gives us \textcolor{red}{infinite} set of linear equations, but the linear space spanned by these integrals is proved to be \textcolor{red}{finite dimensional}.
So, by creating new identities of different $\bold{n}$, we can eventually create enough number of linear equations to solve.
A solution is called \textcolor{red}{Master Integrals}; any integral can be represented as a finite linear combination of MI's.

A well-known public code is Reduze.
More recent implementation is called Kira, which eliminates linear dependent equations before it solves system.
Since linear dependent equations carry no new information toward the system, eliminating such dependent subset can reduce the computational space and time.




\end{document}