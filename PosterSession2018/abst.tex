% LaTex Compiler

\documentclass[10pt]{article}
\usepackage[usenames]{color} %used for font color
\usepackage{amssymb} %maths
\usepackage{amsmath} %maths
\usepackage[utf8]{inputenc} %useful to type directly diacritic characters
\usepackage{tikz-feynman}\begin{document}

\textbf{\textcolor{blue}{Abstract}}

\textcolor{red}{Scattering amplitudes} in quantum field theories allow us to compare the phenomenological prediction of theoretical models with the measurement data at collider experiments.
The study of scattering amplitudes, in terms of their symmetries and analytic properties, can provide a framework to develop techniques and efficient algorithms to evaluate cross sections.
To evaluate higher-order amplitudes, a computational technique called Integration-By-Parts Reduction is at present an unavoidable step.
In this process, a large set of linear relations between integrals is generated; by solving them, we can get a set Master integrals, which are necessary ingredients for our calculations.
In our poster, we describe some interesting representations for this formalism and their applications to generate Integration-By-Parts relations.

Within these representations, integrals are fully characterized by a single polynomial, and the Integration-By-Parts relations becomes a polynomial identity.

{\footnotesize This work was done under the supervision of \textcolor{magenta}{Prof. A. Ferroglia} and \textcolor{magenta}{Prof. G. Ossola.}}

\end{document}