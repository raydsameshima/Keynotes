% LaTex Compiler

\documentclass[10pt]{article}
\usepackage[usenames]{color} %used for font color
\usepackage{amssymb} %maths
\usepackage{amsmath} %maths
\usepackage[utf8]{inputenc} %useful to type directly diacritic characters
\usepackage{tikz-feynman}

\begin{document}

\textbf{\textcolor{blue}{Baikov parametrization}}

Under the integration variable change
\begin{align*}
(l_1, \cdots, l_L) \mapsto (D_1, \cdots, D_N),
\end{align*}
we have
\begin{align*}
I(\vec{n}) \sim \int \frac{d D_1 \cdots dD_N}{D_1^{n_1} \cdots D_N^{n_N}} P^{\frac{d-L-E-1}{2}}
\end{align*}
where $P$ is the Jacobi determinant of this variable change
\begin{align}
\nonumber
P &= \det \left[ q_i \cdot q_j \right] (D_1,\cdots, D_N)
\end{align}
that is, the determinant of scalar products expressed by denominators, and this $P$ is called  the \textcolor{red}{Baikov polynomial}.
The integration domain is determined by the zeros of $P$.

$P$ and the integration domain do not depend on $n_1, \cdots, n_N$, so the family of integrals are characterized by a polynomial $P$.

\end{document}